\chapter{Estudio de soluciones existentes}
Durante la primera fase de este proyecto se realizó una investigación y análisis de herramientas disponibles en el mercado que tuvieran características y funcionalidades similares a las que ofrece DBCase.\\

En este proceso de estudio se han investigado herramientas que permiten la realización de diagramas entidad relación, analizando las mecánicas que ofrecen para la creación de los mismos.\\

También se han investigado herramientas que permitieran la traducción de un esquema conceptual a un código ejecutable por un gestor de base de datos, analizando el flujo de trabajo llevado a cabo por la herramienta.
\section{Draw.io}

Draw.io \cite{drawio} es una aplicación web de código abierto que permite realizar de manera sencilla todo tipo de diagramas siendo una de las herramientas más usadas a nivel mundial para este cometido.\\

Draw.io proporciona una gran cantidad de distintos elementos y formas que pueden ser usadas en el diagrama y proporciona una gran versatilidad y capacidad de personalización sin dejar de ser una herramienta muy intuitiva.\\

Con respecto al proyecto DBCase la herramienta ofrece soluciones muy interesantes en cuanto a la fase de diseño del diagrama entidad relación.\\

\begin{itemize}
    \item Dispone de una barra lateral de elementos y una barra superior de herramientas, mostrando al usuario todos los elementos disponibles así como todas las acciones realizables por el usuario en un solo vistazo.
    \item Permite guardar y abrir proyectos, tanto desde el sistema de ficheros como desde otros sistemas de almacenamientos en la nube.
    \item Permite tener varios proyectos abiertos al mismo tiempo.
    \item El establecimiento de conexiones entre elementos se puede realizar de forma gráfica.
    \item Ofrece la posibilidad de cambiar el tema de la interfaz.
\end{itemize}
\section{Smart Draw}
Smart Draw \cite{smartdraw} es una herramienta de software privativo que permite realizar múltiples tipos de diagrama. Cuenta con una versión web y otra de escritorio para los principales sistemas operativos. Está enfocada principalmente en diagramas de ingeniería y arquitectura y cuenta con un módulo específico para la creación de diagramas entidad relación muy completo.\\

En cuanto al espacio de trabajo que proporciona es muy similar al ofrecido por la herramienta Draw.io (posee una barra lateral de elementos y otra superior de herramientas, permite guardar y abrir proyectos y trabajar con varios al mismo tiempo).\\

Otro punto a destacar es la facilidad de conectar y crear elementos en el diagrama. Llevando el cursor al borde de un elemento permite crear otro idéntico directamente conectado. De esta forma se simplifica enormemente el proceso de conexión de elementos. Esta función es interesante y podría extrapolarse para añadir nuevas entidades a una relación. Esta funcionalidad es complementaria a la de establecer conexiones de forma gráfica, que también ofrece Draw.io.
\section{EDRPLus}
EDRPlus \cite{edrplus} es una aplicación web de uso gratuito, que está específicamente diseñada para la creación de diagramas de bases de datos. La herramienta da la opción de crear diagramas entidad relación y esquemas relacionales. Además la aplicación permite traducir el diagrama entidad relación directamente a código sql.\\

La herramienta permite guardar proyectos y, mediante el inicio de sesión subir y acceder a tus proyectos guardados en el servidor, sacando un gran provecho al hecho de ser una aplicación web.\\

Como vemos, la herramienta permite realizar las dos funcionalidades principales que pretende DBCase.\\

\begin{enumerate}
    \item En cuanto al espacio de trabajo proporcionado para el diseño del diagrama entidad relación, la herramienta es, en general menos potente e intuitiva que las vistas anteriormente. La creación de elementos y conexiones se realiza mediante un panel superior en el que aparecen todos los elementos disponibles (atributos, entidades, relaciones y etiquetas), y todas las acciones disponibles (establecer una conexión, seleccionar, eliminar, deshacer y rehacer) en una sola fila de botones de texto, lo que resulta confuso.\\
    
    Al tener un elemento seleccionado aparece un panel de opciones en la parte derecha de la pantalla que permite modificar todas las opciones del mismo (nombre, tipo, cardinalidad en el caso de relaciones etc). Esto parece una buena solución aunque puede dar pie a que se queden elementos sin modificar, ya que en ningún momento obliga al usuario ni siquiera a establecer un nombre para el elemento.
    
    \item La herramienta no comprueba en ningún momento la corrección del diagrama. Puedes crear atributos sueltos o entidades débiles sin entidad fuerte, en este sentido es menos potente de lo que ya ofrece DBCase.\\
    
    \item La herramienta no ofrece la opción de generar el modelo relacional.
    
    \item Para generar el código SQL, el usuario debe salir del panel de edición e ir al listado de sus documentos, solo desde ahí y mediante un menú desplegable se da la posibilidad de generarlo, lo cual resulta muy poco intuitivo.
    \item No dispone de atajos de teclado.
\end{enumerate}
\section{SqlDBM}
\section{Conclusiones del estudio}