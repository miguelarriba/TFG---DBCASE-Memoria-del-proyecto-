\chapter*{Resumen}%[SPA-ENG]
%%
\textit{[ESP]}\\

\textbf{DBCASE 2.0} ofrece una manera fácil de construir bases de datos relacionales a partir del diseño de un diagrama entidad-relación, la herramienta permite generar al usuario el modelo lógico y físico a partir del diagrama, así como ejecutar directamente el código generado en un DBMS.\\

El principal objetivo de la herramienta es \textbf{guiar} al usuario \textbf{en el proceso de diseño} de una base de datos relacional pasando por todas sus fases. La herramienta está pensada para que los usuarios experimenten con distintos diseños de forma que les permita adquirir y asentar conocimientos.\\

DBCASE está pensada desde un principio como un programa ligero y \textbf{multiplataforma}, lo que permite una gran versatilidad para ser utilizada en cualquier tipo de entorno.

A partir de un diagrama entidad-relación creado por un usuario, la aplicación es capaz de generar el modelo lógico del diseño, así como el modelo físico adaptado a tres gestores de bases de datos distintos: MySQL, ORACLE y ACCESS.\\

\textit{[ENG]}\\

\textbf{DBCASE 2.0} offers an easy way to build relational databases from the design of an entity-relationship diagram, the tool allows the user to generate the logical and physical model from the diagram, as well as directly execute the code generated in a DBMS.\\

The main goal of the tool is to guide the user in the process of designing a relational database through all its phases. The tool is designed for users to experiment with different designs in a way that allows them to acquire and establish knowledge.\\

DBCASE is designed from the beginning as a lightweight and cross-platform program, which allows great versatility to be used in any type of environment.\\

From an entity-relationship diagram created by a user, the application is capable of generating its logical model, as well as the physical model adapted to three different database managers: MySQL, ORACLE and ACCESS.
%%
\section*{Palabras Clave}
Bases de datos | Java | Diagrama Entidad Relación | Modelo Relacional | SQL