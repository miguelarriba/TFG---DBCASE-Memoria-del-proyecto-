\chapter{Resumen}%[SPA-ENG]
%%
\textbf{DBCASE 2.0} ofrece una manera fácil de construir bases de datos relacionales a partir del diseño de un diagrama entidad-relación, la herramienta permite generar al usuario el modelo lógico y físico a partir del diagrama, así como ejecutar directamente el código generado en un gestor de bases de datos.\\

El principal objetivo de la herramienta es \textbf{guiar} al usuario \textbf{en el proceso de diseño} de una base de datos relacional pasando por todas sus fases. La herramienta está pensada para que los usuarios experimenten con distintos diseños de forma que les permita adquirir y asentar conocimientos.\\

DBCASE está pensada desde un principio como un programa ligero y \textbf{multiplataforma}, lo que permite una gran versatilidad para ser utilizada en cualquier tipo de entorno.\\

A partir de un diagrama entidad-relación creado por un usuario, la aplicación es capaz de generar el modelo lógico del diseño, así como el modelo físico adaptado a tres gestores de bases de datos distintos: MySQL, ORACLE y MS ACCESS.\\

La aplicación ofrece una solución completa a todas las fases de diseño de una base de datos, además de estar específicamente diseñada en el proceso de aprendizaje, lo cual la diferencia de las alternativas existentes en el mercado.
%%
\section{Palabras Clave}
Bases de datos | Java | Diagrama Entidad Relación | Modelo Relacional | SQL